\section{Aim Of Research}\label{aor}
\hyphenation{Non-para-metric}
The primary objective of this research is to introduce the Bayesian Nonparametric(BNP) theory to the general framework of Active Leaning. We aim to design a new learning framework enabling the growing of model complexity without a great compromising in accuracy and a growth in cost.
The initial motivation is take advantages of the merits , as for active learning the efficiency and for BNP the flexibility, to make up for their disadvantage, as for active learning low sampling procedure and and for BNP low convergency rate\cite{gershman2012tutorial,escobar1995bayesian,Settles2010}. In this proposal, we will exploit the general problems in this two distinct areas and the  potential performance when combined together.  The main aims of this research consist of the following parts(will be in addressed in detail in later part) 
\begin{enumerate}
\item{\textbf{Develop a new sampling methods For Dirichlet Process Mixture Model}}\\
 Sampling strategy lies in the core of BNP inference scheme. Most approaches select either informative or representative unlabeled instances. But it is usually challenging to find the querying instances that are both informative and representative. Thus in this research, a new approach is to be proposed to provide a systematic way to select samples having both features, and the effectiveness will be validated.
 
\item{\textbf{Design a distributed metric learning methods on big data set }}\\

Distance metric learning is fundatmental problem in data mining. Many representative data mining algorithms rely on the underlying distance metric for correctly measurin grelations among input data. But metric learning performs poor in efficiency on large data set as the computation procedure required larege compuation source. In this research, we will try to redesign the metric learning algorithm under distributed framework and apply it to large data set.

\item{\textbf{Application of metric learning on multi-modal }}\\

Multi-modal data is dramatically increasing with the fast growth of social media. Learning a good distance measure for data with multiple modalities is of vital importance for many applications, in- cluding retrieval, clustering, classification and rec- ommendation. In this research, we will conduct research on the multi-modal distance metric learning.

 
\end{enumerate}

