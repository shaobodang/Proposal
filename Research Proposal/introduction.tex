\section{Introduction}
%bigdata
\indent Data Mining has been an heating area in Machine Learning. It is especially important in the information age since the seemingly meaningless data often containing potential opportunity, as can be seen from large Internet companies such as Google, Facebook and Amazon \cite{mcafee2012big}on their investment on large data processing project. 

The primary task in data mining is to find the patterns, often referred to as models, in the data to guide the computer to process the similar data automatically, saving the cost of both time and expense. In terms of kind of data to be mined, there are two major types of functions in this area\cite{bishop2006pattern}:
\begin{itemize}
\item Descriptive: dealing with general properties of data in the database to find the underlying generative patten to reproduce new samples.
\item Classification and Prediction: finding a model that describes the data classes or concepts which could be used to predict the class of objects with unkown class label.  
\end{itemize}
Both of the green acquisition of the above two types of functions \colorbox{green}{contain an}important training process to get an estimated parameters of the proposed latent model on a data set with known characteristics(including its feature and class label) called training set. The \colorbox{green}{desired function or model} is trained on the traing set to obtain an desired level of evaluation criterion. Often, the training accuracy is  in positive correlation to the size of the training set\colorbox{green}{\cite{dietterich2000ensemble}}. 
%current difficulties and need for this research


However, there are always scenarios, in which vast quantities of unlabeled data(such as images and videos on the web, speeches recorded via microphones and so on) are easy to collect while their labels are not, regardless of the type of the learning scheme(supervised or unsupervised)\cite{Settles2010,druck2009active}.By regular supervised learning, their might be a compromise in accuracy. Moreover, the computation complexity will grow as the size of the data set grows. This could be a disaster in data mining related area. 
% %the drawback of supervised learning

%proposed solution
Active learning is motivated on this scenario\colorbox{green}{\cite{Settles2010}}. The main idea is to query the labels of a subset in which the samples are especially informative to obtain an accurate learner at lower cost but reaching a higher or at least similar performance than the regular supervised learning(requiring large training set) algorithms. Also to cope with the large data set containing multiple models, bayesian non-parametric models is introduces to flexibly adjust itself to better fit the data set\cite{walker1999bayesian,ghosh1982nonparametric}. 

