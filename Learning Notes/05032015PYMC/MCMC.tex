\documentclass[a4paper,10pt]{article}
\usepackage[utf8]{inputenc}
\usepackage{amsmath,amsthm}
%opening
\title{MCMC Notes}
\author{Shaobo}

\begin{document}

\maketitle
\section{Markov Chain}
A Markov chain is a special type of stochastic process. The standard definition of a stochastic process is an ordered collection of 
random variables:



where $t$ is frequently (but not necessarily) a time index. If we think of $X_t$ as a state $X$ at time $t$, and invoke the following 
dependence condition on each state:



then the stochastic process is known as a Markov chain. This conditioning specifies that the future depends on the current state, but not 
past states. Thus, the Markov chain wanders about the state space,remembering only where it has just been in the last time step. The 
collection of transition probabilities is sometimes called a transition matrix when dealing with discrete states, or more generally, a
transition kernel.

In the context of Markov chain Monte Carlo, it is useful to think of the Markovian property as �mild non-independence MCMC allows us to 
indirectly generate independent samples from a particular posterior distribution.

\section{Gibbs Sampling}
\begin{enumerate}
 \item  If a posterior has $k$ parameters to be estimated, we may condition each parameter on current values of the other $k-1$ parameters,
 and sample from the resultant distributional form (usually easier), and repeat this operation on the other parameters in turn. This procedure
 generates samples from the posterior distribution. Note that we have now combined Markov chains (conditional independence) and Monte Carlo
 techniques (estimation by simulation) to yield Markov Chain Monte Carlo.
\end{enumerate}


\section{PYMC}
PyMC is a python module that implements Bayesian statistical models and
fitting algorithms, including Markov chain Monte Carlo. Its flexibility
and extensibility make it applicable to a large suite of problems. Along
with core sampling functionality, PyMC includes methods for summarizing
output, plotting, goodness-of-fit and convergence diagnostics.

PyMC provides functionalities to make Bayesian analysis as painless as
possible. Here is a short list of some of its features:
\begin{itemize}
 \item Fits Bayesian statistical models with Markov chain Monte Carlo and
    other algorithms.
\item Includes a large suite of well-documented statistical distributions.
\item   Uses NumPy for numerics wherever possible.
\item   Includes a module for modeling Gaussian processes.
\item   Sampling loops can be paused and tuned manually, or saved and
    restarted later.
\item   Creates summaries including tables and plots.
\item   Traces can be saved to the disk as plain text, Python pickles,
    SQLite or MySQL database, or hdf5 archives.
\item   Several convergence diagnostics are available.
\item   Extensible: easily incorporates custom step methods and unusual
    probability distributions.
\item   MCMC loops can be embedded in larger programs, and results can be
    analyzed with the full power of Python.
\end{itemize}
Bayesian inference begins with specification of a probability model relating unknown variables to data. PyMC provides three basic
building blocks for Bayesian probability models: \textbf{Stochastic, Deterministic and Potential}.
\begin{enumerate}
 \item The Stochastic Class
A stochastic variable has the following primary attributes:
value : The variable's current value.
logp : The log-probability of the variable's current value given the values of its parents.
 \item The Deterministic Class
 \item The Potential Class
\end{enumerate}

\end{document}
